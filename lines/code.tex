\documentclass[12pt]{article}
\usepackage[hidelinks]{hyperref}
\usepackage{graphicx}
%\documentclass[journal,12pt,twocolumn]{IEEEtran}
\usepackage[none]{hyphenat}
\usepackage{graphicx}
\usepackage{listings}
\usepackage[english]{babel}
\usepackage{graphicx}
\usepackage{caption} 
\usepackage{hyperref}
\usepackage{booktabs}
\def\inputGnumericTable{}
\usepackage{color}                                            %%
\usepackage{array}                                            %%
\usepackage{longtable}                                        %%
\usepackage{calc}                                             %%
\usepackage{multirow}                                         %%
\usepackage{hhline}                                           %%
\usepackage{ifthen}
\usepackage{array}
\usepackage{amsmath}   % for having text in math mode
\usepackage{listings}
\usepackage{caption}
\usepackage{refstyle}
\graphicspath{{/sdcard/Download/vectors/tables}}
\lstset{
language=tex,
frame=single, 
breaklines=true
}
  
%Following 2 lines were added to remove the blank page at the beginning
\usepackage{atbegshi}% http://ctan.org/pkg/atbegshi
\AtBeginDocument{\AtBeginShipoutNext{\AtBeginShipoutDiscard}}
%


%New macro definitions
\newcommand{\mydet}[1]{\ensuremath{\begin{vmatrix}#1\end{vmatrix}}}
\providecommand{\brak}[1]{\ensuremath{\left(#1\right)}}
\providecommand{\norm}[1]{\left\lVert#1\right\rVert}
\newcommand{\solution}{\noindent \textbf{Solution: }}
\newcommand{\myvec}[1]{\ensuremath{\begin{pmatrix}#1\end{pmatrix}}}
\let\vec\mathbf


\begin{document}

\begin{center}
\title{\textbf{VECTOR ALGEBRA}}
\date{\vspace{-5ex}} %Not to print date automatically
\maketitle
\end{center}

\setcounter{page}{1}

\section*{12$^{th}$ Maths - Chapter 10}

This is Problem-8 from Exercise 5.8

 
\begin{enumerate}

	\item Show that the points A $\myvec{1\\-2\\-8}$, B $\myvec{5\\0\\-2}$ and C $\myvec{11\\3\\7}$ are collinear, and find the ratio in which B divides AC.\\


\solution \\The input parameters for this problem are available in Table \ref{Table-1}
\begin{table}[ht!]
\input{tables/table.tex}
\caption{}
\label{Table-1}	

\end{table}

		$\vec{A}$, $\vec{B}$ and $\vec{C}$ are collinear if,

\begin{align}
	\norm{\overrightarrow{AB}}+\norm{\overrightarrow{BC}} &= 
	\norm{\overrightarrow{AC}}
\end{align}
	Here,\\

		$\overrightarrow{AB}$ can be expressed as follows\\
		\begin{align}
			\overrightarrow{AB} &=
			(5-1)\hat{i}+(0+2)\hat{j}+(-2+8)\hat{k} =
		4\hat{i}+2\hat{j}+6\hat{k}
		\end{align}
	of magnitude  \\
		\begin{align} 
		\norm{\overrightarrow{AB}} &=
			\sqrt{16+4+36} =
		2\sqrt{14}
		\end{align}

		Similarly, $\overrightarrow{BC}$ and $\overrightarrow{AC}$ can be expressed as\\

		\begin{align}
			\overrightarrow{BC} &=
			(11-5)\hat{i}+(3-0)\hat{j}+(7+2)\hat{k} =
		6\hat{i}+3\hat{j}+9\hat{k}
		\end{align}

		\begin{align}
			\overrightarrow{AC} &=
			(11-1)\hat{i}+(3+2)\hat{j}+(7+8)\hat{k} =
		10\hat{i}+5\hat{j}+15\hat{k}
		\end{align}
		
		of magnitudes\\
		
		\begin{align} 
		\norm{\overrightarrow{BC}} &=
			\sqrt{36+9+81} =
		3\sqrt{14}
		\end{align}


		\begin{align} 
		\norm{\overrightarrow{AC}} &=
			\sqrt{100+25+225} =
		5\sqrt{14}
		\end{align}
		
		where\\

		\begin{align}
			\norm{\overrightarrow{AB}}+\norm{\overrightarrow{BC}} &= 
			2\sqrt{14}+3\sqrt{14} = 5\sqrt{14}
		\end{align}

		Thus,

		\begin{align}
			\norm{\overrightarrow{AB}}+\norm{\overrightarrow{BC}} &= 
			\norm{\overrightarrow{AC}}
		\end{align}

		Let $\vec{B}$ divide $\vec{AC}$ in k:1 then,\\
		
		\begin{align}
			\myvec{5\\0\\-2} &=
			\frac{k\myvec{11\\3\\7}+1\myvec{1\\-2\\-8}}{k+1} =
			\frac{\myvec{11k+1,3k-2,7k-8}}{k+1}
		\end{align}
		
		$\implies k=2/3 \implies k:1 \implies 2:3$ \\
		
		Hence, $\vec{B}$ divides $\vec{AC}$ in 2:3.


\end{enumerate}
\end{document}





