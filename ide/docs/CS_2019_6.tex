\documentclass[a4paper,11pt,twocolumn]{article}
\usepackage[a4paper,left=1.5cm,right=1cm,top=2cm,bottom=2cm]{geometry}
\usepackage{setspace}
\usepackage{gensymb}
\usepackage{caption}
\usepackage{graphicx}
\usepackage{tabularx}
\usepackage{lmodern}
\usepackage{watermark} 
\usepackage{lipsum}
\usepackage{xcolor}
\usepackage{listings}
\usepackage{graphicx}
\usepackage{enumitem}
\usepackage{mathtools}
\usepackage{titlesec}
\usepackage[utf8]{inputenc}
\usepackage{fontenc}
\usepackage{harvard}
\usepackage{amsfonts}
\graphicspath{{/storage/emulated/0/Download/FWC/Latex/figs}}
\usepackage[colorlinks,linkcolor={black},citecolor={blue!80!black},urlcolor={blue!80!black}]{hyperref}
\thiswatermark{\centering \put(-15,-100.0){\includegraphics[scale=0.3]{figs/logo.jpg}} }
\title{\textbf{\textsc{VERIFICATION OF BOOLEAN IDENTITIES}}}
\author{\textbf{\textit{\teflipflopxtbf{MADHU LATHA ADDANKI (FWC22129)}}}}
\begin{document}

\date{}
\maketitle
\tableofcontents


\section{PROBLEM}
\textbf{(GATE CS-2019)}
\textbf{Q.6} Which one the following is not a valid identity?
\begin{enumerate}[label=(\Alph*)]
	\item $ (x\oplus y)\oplus z = x\oplus (y\oplus z)$
	\item $ (x + y)\oplus z = x\oplus (y + z)$
	\item $ x\oplus y = x + y, if xy = 0$
	\item $ x\oplus y = (xy + x'y')'$
\end{enumerate}
\bigskip

\section{COMPONENTS}
	\begin{tabularx}{0.45\textwidth} {  
  | >{\centering\arraybackslash}X  
  | >{\centering\arraybackslash}X  
  | >{\centering\arraybackslash}X | } 
\hline 
\textbf{Component} &  \textbf{Value} & \textbf{Quantity}\\ 
\hline 
Arduino & UNO & 1 \\   
\hline 
Bread board & - & 1 \\ 
\hline 
IC & 7447 & 1 \\
\hline
Jumper wires & M-M & 20 \\ 
\hline 
SevenSegment Display & - & 1\\ 
\hline 
Resistor & 150ohms & 1\\ 
\hline 
\end{tabularx}
\bigskip

\section{INTRODUCTION}
\paragraph{}
	An "identity" is merely a relationship that is always true, regardless of the values that any variables involved might take on; similar to laws or properties. Many of these can be analogous to normal multiplication and addition, particularly when the symbols {0,1} are used for {FALSE, TRUE}. 
\bigskip 

\section{TRUTH TABLE}
The Truth Table for the above identities is ass follows:
\begin{enumerate}[label=\textbf{(\Alph*})]
	\item \textbf{$  (x\oplus y)\oplus z = x\oplus(y\oplus z)$} \\
where $Y1=(x\oplus y)\oplus z,Y2=x\oplus(y\oplus z)$\\
\bigskip
\begin{table}[ht!]
	\centering
\begin{tabular}{ |c |c |c |c |c |c |} 
\hline 
\newline 
	\textbf{x} & \textbf{y} & \textbf{z} & \textbf{Y1} & \textbf{Y2} & \textbf{Y1==Y2} \\ 
\hline  
	0 & 0 & 0 &0 &0 &1\\   
	0 & 0 & 1 &1 &1 &1\\  
	0 & 1 & 0 &1 &1 &1\\  
	0 & 1 & 1 &0 &0 &1\\  
	1 & 0 & 0 &1 &1 &1\\  
	1 & 0 & 1 &0 &0 &1\\  
	1 & 1 & 0 &0 &0 &1\\  
	1 & 1 & 1 &1 &1 &1\\  
\hline 
\end{tabular}
	\caption{}
\end{table}
\bigskip
\bigskip
	
	\item \textbf{$(x+y)\oplus z=x\oplus(y+z)$}\\
where $Y1=(x+y)\oplus z, Y2=x\oplus(y+z)$\\
\bigskip
\begin{table}[ht!]
	\centering
\begin{tabular}{ |c |c |c |c |c |c |} 
\hline 
\newline 
	\textbf{x} & \textbf{y} & \textbf{z} & \textbf{Y1} & \textbf{Y2} & \textbf{Y1==Y2}\\
\hline  
	0 & 0 & 0 &0 &0 &1\\   
	0 & 0 & 1 &1 &1 &1\\ 
	0 & 1 & 0 &1 &1 &1\\  
	0 & 1 & 1 &0 &1 &0\\  
	1 & 0 & 0 &1 &1 &1\\  
	1 & 0 & 1 &0 &0 &1\\  
	1 & 1 & 0 &1 &0 &0\\  
	1 & 1 & 1 &0 &0 &1\\  
\hline 
\end{tabular}
\caption{}
\end{table}
\bigskip

	\item \textbf{$x\oplus y = x + y, if xy = 0 $} \\
where $Y1=x\oplus y=x+y, if xy=0$\\
\bigskip
\begin{table}[ht!]
	\centering
\begin{tabular}{ |c |c |c |c |c |} 
\hline 
\newline 
\textbf{x} & \textbf{y} & \textbf{Y1} & \textbf{Y2} & \textbf{Y1==Y2} \\ 
\hline 
	0 & 0 & 0 &0 &1\\   
	0 & 1 & 1 &1 &1\\  
	1 & 0 & 1 &1 &1\\    
 \hline 
 \end{tabular}
	\caption{}
\end{table}
\bigskip

	\item \textbf{$ x\oplus y = (xy + x'y')'$}\\
		where $(xy+x'y')'=(x'+y')(x+y)$\\
		      $=x\oplus y$\\
The Truth Table for $x\oplus y$ is as follows:\\
\begin{table}[ht!]
	\centering
	\begin{tabular}{ |c |c |c |} 
\hline 
\newline 
\textbf{x} & \textbf{y} & \textbf{$x\oplus y$} \\ 
\hline 
 0 & 0 & 0\\   
 0 & 1 & 1\\  
 1 & 0 & 1\\  
 1 & 1 & 0\\  
 \hline 
 \end{tabular}
	\caption{}
\end{table}
\bigskip

\paragraph{}
	Here, Except \textbf{(B)} identity all other identies are valid according to the mentioned truth tables.
\end{enumerate}
\bigskip

\section{ARDUINO CONNECTIONS}

1) The connections between IC 7447 and Seven Segment Display are as follows:
\begin{table}[ht!] 
    \centering 
    \begin{tabular}{|c|c|c|c|c|c|c|c|} 
    \hline 
       7447& $\bar a$&$\bar b$&$\bar c$&$\bar d$&$\bar e$&$\bar f$&$\bar g  $ \\ 
       \hline 
    DISPLAY& a&b&c&d&e&f&g \\ 
    \hline 
    \end{tabular} 
    \caption{} 
\end{table} 
\\

2) The connections between IC 7447 and Arduino are as fllows:
\begin{table}[ht!] 
    \centering 
    \begin{tabular}{|c|c|} 
    \hline 
        IC7447&A  \\ 
         \hline 
         ARDUINO&2 \\ 
         \hline 
    \end{tabular} 
\caption{} 
\end{table} 
\\

3) The inputs \textbf{x,y,z} here are connected to Arduino D5,D6,D7 pins.\\

4) The values for these inputs are conncted either to GND or 5V according to the truth table.\\
\section{CODE}
\paragraph{}
	The arduino code can be downloaded from the below link.
\begin{center} 
\fbox{\parbox{8.5cm}{\url{https://github.com/madhu-addanki/FWC/tree/main/ide }}} 
\end{center}


\end{document}

